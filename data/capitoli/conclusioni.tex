\section{Conclusioni}
Nei capitoli precedenti, attraverso gli esempi osservati, abbiamo approcciato lo stesso problema in diversi modi, utilizzando più di un metodo per migliorare la fruizione all'utente e lo sforzo mnemonico del sistema.\par

\subsection{Approccio classico}\label{conclusioni_caso1}
Applicare direttamente tutti i principi dell'algoritmo di Hough, abbiamo visto essere inefficace agli obiettivi che ci siamo posti. Questo perché l'algoritmo originale prevede di avere una dimensione per ogni parametro della curva nell'immagine; ciò porta ad utilizzare un tensore quadridimensionale.\par
La rappresentazione di cinque dimensioni in modo comprensibile all'utente, cioè che non induca in errori di interpretazione, è pressoché impossibile.\par
Oltre questo, mantenere in memoria una quantità di dati abnorme e, su questi, ricercare massimi locali all'interno del tensore è anche computazionalmente complesso.\par
È di facile comprensione la grande quantità di valori presenti, nel caso si intenda analizzare un'immagine Full-HD\footnote{Le immagini Full HD hanno una risoluzione pari a 1920x1080 pixel.} per un discreto numero di curve. Se ipotiziamo valori di $a$ in un range $[0.1,\ 2]$ con passo di $0.05$ e, $d$ in un range $[10,\ \text{<numero di righe>}]$ con passo $2$, otteniamo facilmente:
$$\text{Numero di valori} = \underbrace{1920\times 1080}_{\text{dim. immagine}}\times\underbrace{\frac{2-0.1}{0.05}}_{\text{valori di a}}\times\underbrace{\frac{1080-10}{2}}_{\text{valori di d}}\approx 20\times10^9$$

\subsection{Approccio classico limitato}\label{conclusioni_caso2}
Successivamente abbiamo provato ad arginare questi due macro problemi fissando una dimensione. Abbiamo deciso di fissare $d$, in quanto è possibile che l'ampiezza delle curve sia dovuta ad una costante fisica: nel caso di satelliti non geostazionari, potrebbe essere la velocità relativa del satellite rispetto alla terra.\par
Per quanto riguarda il numero di valori, questo approccio permette di limitarli relativamente.\par
Nell'esempio di prima, possiamo togliere l'ultimo rapporto, lasciando:
$$\text{Numero di valori} = \underbrace{1920\times 1080}_{\text{dim. immagine}}\times\underbrace{\frac{2-0.10}{0.05}}_{\text{valori di a}}\times \underbrace{1}_{\text{valori di d}}\approx 37\times10^6$$
Questo metodo, a differenza degli altri due, riduce anche le iterazioni necessarie per il computo delle occorrenze.\par
Mentre per quanto riguarda la visualizzazione, abbiamo trovato un metodo che, oltre alle tre dimensioni classiche, usa i colori.\par
Tuttavia, questo metodo crea un grafico che non è esattamente di immediata lettura e questo può portare a notevoli errori. Per avere un'informativa completa è necessario traslare i vari piani di taglio per visualizzare tutte le possibili combinazioni, questo richiede tempo e conoscenza da parte dell'utente.\par

\subsection{Approccio innovativo}
Il nostro approccio coniuga la semplicità rappresentativa del risultato con una ridotta necessità di memoria.\par
Questo perché la struttura che accoglie i dati è una semplice matrice bidimensionale. Essa è rappresentante dell'immagine, quindi abbiamo come coordinate della matrice quelle dei pixel.\par
Come abbiamo visto nell'\hyperref[lst:ostinelli]{algoritmo}, il sistema salva i dati inerenti alla curva testata e al numero di occorrenze che questa ha generato, solo nel caso in cui non ci sia stata una curva già testata con maggior numero di occorrenze.\par
Facendo questo, ci accorgiamo che il numero di dati salvati è leggermente minore del \hyperref[conclusioni_caso2]{secondo caso} e sensibilmente del \hyperref[conclusioni_caso1]{primo}.
$$\text{Numero di valori} = \underbrace{1920\times 1080}_{\text{dim. immagine}}\times\underbrace{3}_{\text{num. parametri}}\approx 6\times10^6$$

\subsubsection{Fragilità principale}
Come abbiamo visto nel \hyperref[fig:sintentica_ostinelli_6]{sesto esempio} delle immagini sintetiche e nel \hyperref[fig:reale_ostinelli_2]{secondo esempio} delle immagini reali, il nostro algoritmo, in caso di più curve ad S presenti nell'immagine, rischia di segnalare la presenza di curve in realtà inesistenti. Questi falsi positivi, sono dati dalla sovrapposizione immaginaria di una curva più grande con le due più piccole.\par
Questo problema è intrinseco alla nostra scelta di rappresentare in uno stesso grafico a prescindere dai parametri delle stesse.