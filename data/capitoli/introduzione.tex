\section{Introduzione alla trasformata di Hough}
La trasformata di Hough è una tecnica utilizzata per rilevare forme geometriche, in particolare linee, cerchi o altri tipi di curve, in un'immagine. Partiamo dal caso più semplice: la trasformata di Hough per linee. Supponiamo di avere un'immagine binaria in cui desideriamo individuare le linee presenti. La trasformata di Hough per linee si basa sull'idea che ogni linea può essere rappresentata da un punto nel parametro spazio. Questo spazio è definito dalle due coordinate $(\rho,\ \theta)$, dove $\rho$ è la distanza dalla linea all'origine e $\theta$ è l'angolo tra la linea e l'asse $x$. \par
Il processo di trasformazione di Hough coinvolge la scansione di ciascun punto nell'immagine binaria. Per ogni punto nero, viene eseguita una scansione attraverso il parametro spazio $(\rho,\ \theta)$. Per ogni valore di $\rho$ e $\theta$, viene incrementato un accumulatore.\par
Una volta completata la scansione, i picchi nell'accumulatore indicano le linee rilevate nell'immagine. Tuttavia, è necessario impostare una soglia critica sull'accumulatore nel caso in cui si voglia selezionare solo i picchi significativi; ad esempio, rette che abbiano uno o più segmenti che insieme siano abbastanza lunghi.\par
La trasformata di Hough per linee è ampiamente utilizzata in applicazioni di visione artificiale ed elaborazione delle immagini, utili in settori come il riconoscimento di forme e oggetti, l'elaborazione di immagini mediche, la visione robotica, il controllo di qualità industriale, la guida autonoma e l'analisi di documenti.\par
La trasformata di Hough è stata introdotta per la prima volta da Richard Duda e Peter Hart nel 1972 \cite{houghtransform}.\par
Nella pagina successiva,  vediamo un semplice \hyperref[lst:houghlineare]{esempio di pseudocodice} per la visualizzazione di tale trasformata lineare in 3 dimensioni.

\subsection{Altri tipi noti: la trasformata per i cerchi}
Nel caso l'obiettivo sia trovare cerchi all'interno di un'immagine, organizziamo lo spazio in questo modo. Ogni pixel dell'immagine viene rappresentato come un punto nello spazio dei parametri Hough, dove i parametri sono le coordinate del centro $(x,\ y)$ e il raggio $r$. Utilizzando un accumulatore tridimensionale, vengono accumulati i voti per ogni possibile centro del cerchio e raggio, consentendo l'individuazione dei cerchi mediante l'identificazione dei picchi nell'accumulatore. \par
Una volta individuati i picchi nell'accumulatore, che rappresentano i centri e i raggi dei cerchi rilevati, è possibile applicare ulteriori criteri di filtraggio, come la soglia o altri criteri di selezione dei picchi, per raffinare la selezione dei cerchi.

\newpage
\lstinputlisting[
    style=matlabStyle,
    caption={Codice MATLAB per la trasformata di Hough lineare},
    label={lst:houghlineare}
]{codici/trasformataDiHoughLineare.m}
\newpage

\subsection{Obiettivo}
L'obiettivo di questo lavoro è sviluppare un approccio innovativo per identificare e tracciare curve a forma di S nelle immagini satellitari. Per raggiungere questo scopo, abbiamo per prima cosa parametrizzato un prototipo di curva ad S, nello specifico una sigmoide, tramite due variabili.\par
Dopodiché, mediante questa parametrizzazione, con la semplice applicazione dei principi standard, siamo arrivati a costruire un tensore\footnote{la nozione di tensore generalizza tutte le strutture definite usualmente in algebra lineare a partire da un singolo spazio vettoriale\cite{wiki:tensore}.} in quattro dimensioni, due che indicassero le coordinate del centro e due per la curva vera e propria, il quale conterrà il numero di occorrenze che tale curva in quel punto ha nell'immagine.\par
Successivamente, abbiamo modificato alcuni principi originali della trasformata per adattarla ad esigenze diverse. Tra i risultati della trasposizione della trasformata di Hough, infatti, c'è il limite abbastanza gravoso della crescita dimensionale.\par
La nostra metodologia, di conseguenza, si distingue perché utilizzeremo una trasformata che converte uno spazio in origine a cinque dimensioni in uno a tre dimensioni, consentendoci di tralasciare dettagli più complessi ma spesso irrilevanti e di gestire più comodamente la visualizzazione del risultato.\par

\subsection{Trasformata generalizzata}
Esiste anche un altro metodo, che enunciamo ma che non tratteremo, di approcciarsi al problema. Questo tralascia l'impostazione analitica per concentrarsi sui modelli di adattamento.\par
Il Generalized Hough Transform (GHT)\cite{ballard1981generalizing}, introdotto da Dana H. Ballard nel 1981, è una modifica della trasformata di Hough che utilizza il principio del riconoscimento dei modelli. Mentre la trasformata di Hough originale era utilizzata per rilevare forme analiticamente definite (come linee, cerchi, ellissi, ecc.), il GHT consente di rilevare oggetti arbitrari descritti tramite un modello. Questa modifica trasforma il problema di individuare un oggetto in un'immagine nel problema di trovare i parametri di trasformazione che mappano il modello nell'immagine.